\documentclass{ipol}

\ipolSetTitle{Four Gigabytes : a Local Installation of Stable Diffusion}

\ipolSetAuthors{First Author \ipolAuthorMark{1},
                Second Author\ipolAuthorMark{2}}

\ipolSetAffiliations{%
\ipolAuthorMark{1} Department, Institution, Country
                   (\texttt{user@server.net})\\
\ipolAuthorMark{2} Department, Institution, Country
                   (\texttt{username@mailserver.edu})}

\ipolPreprintLink{http://www.ipol.im/}



\begin{document}

\begin{ipolAbstract}
The abstract should contain about 100 to 150 words, and should be
identical to the abstract text submitted electronically. An abstract
must be able to stand alone, independent of the paper.  Written in
plain text, it cannot contain citations to the paper’s references or
equations or footnotes and should not, if possible, include special
characters like math notations or greek letters, or hyperlinks. The
abstract must be a single paragraph; multiple parts can be split with
a single line break.
\end{ipolAbstract}

\begin{ipolCode}
The source code section briefly explains what the source code
published with the article contains, all in a single paragraph. For
example: The reviewed source code and documentation for this algorithm
are available from \href{\ipolLink}{the web page of this
  article}. Compilation and usage instruction are included in the
\verb|README.txt| file of the archive.
\end{ipolCode}


\begin{ipolSupp}
The supplementary files section provides explanations about other
files published with the article, all in a single paragraph. Mention
clearly if they are reviewed or not. For example: A reference dataset,
to be used for further comparisons, is provided with the article and
peer-reviewed. A Matlab interface (not reviewed) is also available for
convenience.
\end{ipolSupp}

\ipolKeywords{first, second, third, fourth}


\section{Introduction}



%\bibliographystyle{siam}
%\bibliography{article}

\end{document}
